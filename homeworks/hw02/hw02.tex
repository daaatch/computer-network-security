\documentclass{article}

\title{Cybersecurity - Homework \#2}
\author{Vlad Turno (1835365)}
\usepackage{tcolorbox}
\usepackage{listings}
\usepackage{graphicx}

\begin{document}

\maketitle

\section*{1) Introduction}
In cryptography a block cipher is an algorithm that operates iterating over blocks of bits having a fixed length. In this homework there will be a performance analysis over three different CBC algorithms (AES, Aria and Camellia) all using a symmetric 128-bit key to encrypt and decrypt some files.

\section*{2) Performance Analysis}
With a simple but quite not short C program it is possible to create some files having a specified dimension (1KB, 10KB and 1MB) alongside with a random generated 128-bit key and by using some functions from the OpenSSL library we can encode and decode stuff using the cpu clock to calculate the operation time.\\

\lstset{
  language=C,                     
  numbers=left,                
  stepnumber=1,                       
  numbersep=5pt,                  
  backgroundcolor=\color{white},  
  showspaces=false,               
  showstringspaces=false,         
  showtabs=false,               
  tabsize=2,                      
  captionpos=b,                  
  breaklines=true,               
  breakatwhitespace=true,     
  title=\lstname,             
}
\lstinputlisting{comparator.c}

\section*{3) Observations}
Now we take the results from our C program and we plot them into a chart to visualize better the results. It's easy to note how all the algorithms performs similarly over the small files (1-10kb) but how things change drastically when they work with a slightly bigger file (1mb)\\

\begin{figure}[h]
  \centering
  \includegraphics[width=\textwidth]{Encryption.png}
  \caption{}
  \label{fig:enc}
\end{figure}

\begin{figure}[h]
  \centering
  \includegraphics[width=\textwidth]{Decryption.png}
  \caption{}
  \label{fig:enc}
\end{figure}

\end{document}
