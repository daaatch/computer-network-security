\documentclass{article}
\usepackage[utf8]{inputenc}
\title{Cybersecurity - Homework \#8}
\author{Vlad Turno (1835365)}
\usepackage{listings}
\usepackage{minted}

\begin{document}

\maketitle

\section*{1) Existing Configuration}
I'm considering a Debian Linux machine, so iptables is installed by default. First of all check, if existing, the current configuration via the following command\\

\begin{minted}{bash}
-----------------------------------------------------------------------
$ sudo iptables -L -v -n
-----------------------------------------------------------------------
Chain INPUT (policy ACCEPT 0 packets, 0 bytes)
 pkts bytes target     prot opt in     out     source      destination         

Chain FORWARD (policy ACCEPT 0 packets, 0 bytes)
 pkts bytes target     prot opt in     out     source      destination         

Chain OUTPUT (policy ACCEPT 0 packets, 0 bytes)
 pkts bytes target     prot opt in     out     source      destination   
-----------------------------------------------------------------------
\end{minted}

\section*{2) Possible scenarios}
Default policies can be modified in order to accept or drop traffic in the 3 possible chains\\

\begin{minted}{bash}
-----------------------------------------------------------------------
$ sudo iptables -P INPUT ACCEPT
$ sudo iptables -P FORWARD DROP
$ sudo iptables -P OUTPUT DROP
$ sudo iptables -L -v -n
-----------------------------------------------------------------------
Chain INPUT (policy ACCEPT 44 packets, 6498 bytes)
 pkts bytes target     prot opt in     out     source       destination         

Chain FORWARD (policy DROP 0 packets, 0 bytes)
 pkts bytes target     prot opt in     out     source       destination         

Chain OUTPUT (policy DROP 35 packets, 4236 bytes)
 pkts bytes target     prot opt in     out     source       destination  
-----------------------------------------------------------------------
\end{minted}

\subsection*{2.1) First Scenario}
First thing that could be done is to deny remote shell connections by blocking traffic on SSH port (by default is 22)\\

\begin{minted}{bash}
-----------------------------------------------------------------------
$ sudo iptables -A INPUT -p tcp --dport 22 -j DROP
$ sudo iptables -L -v -n
-----------------------------------------------------------------------
Chain INPUT (policy ACCEPT 48 packets, 8266 bytes)
 pkts bytes target     prot opt in     out     source       destination         
    0     0 DROP       6    --  *      *       0.0.0.0/0    0.0.0.0/0        tcp dpt:22

Chain FORWARD (policy DROP 0 packets, 0 bytes)
 pkts bytes target     prot opt in     out     source       destination         

Chain OUTPUT (policy DROP 265 packets, 39919 bytes)
 pkts bytes target     prot opt in     out     source       destination  
-----------------------------------------------------------------------
\end{minted}

\subsection*{2.2) Second Scenario}
Second thing that could be done is to consent HTTP and HTTPS traffic (ports 80 and 443)\\

\begin{minted}{bash}
-----------------------------------------------------------------------
$ sudo iptables -A FORWARD -p tcp --dport 80 -j ACCEPT
$ sudo iptables -A FORWARD -p tcp --dport 443 -j ACCEPT
$ sudo iptables -A OUTPUT -p tcp --dport 80 -j ACCEPT
$ sudo iptables -A OUTPUT -p tcp --dport 443 -j ACCEPT
$ sudo iptables -L -v -n
-----------------------------------------------------------------------
Chain INPUT (policy ACCEPT 109 packets, 15347 bytes)
 pkts bytes target     prot opt in     out     source         destination         
    0     0 DROP       6    --  *      *       0.0.0.0/0      0.0.0.0/0      tcp dpt:22
    0     0 ACCEPT     6    --  *      *       0.0.0.0/0      0.0.0.0/0      tcp dpt:80
    0     0 ACCEPT     6    --  *      *       0.0.0.0/0      0.0.0.0/0      tcp dpt:443
    4   260            0    --  *      *       0.0.0.0/0      0.0.0.0/0           

Chain FORWARD (policy DROP 0 packets, 0 bytes)
 pkts bytes target     prot opt in     out     source         destination         
    0     0 ACCEPT     6    --  *      *       0.0.0.0/0      0.0.0.0/0      tcp dpt:80
    0     0 ACCEPT     6    --  *      *       0.0.0.0/0      0.0.0.0/0      tcp dpt:443

Chain OUTPUT (policy DROP 5499 packets, 382K bytes)
 pkts bytes target     prot opt in     out     source         destination         
    0     0 ACCEPT     6    --  *      *       0.0.0.0/0      0.0.0.0/0      tcp dpt:80
    0     0 ACCEPT     6    --  *      *       0.0.0.0/0      0.0.0.0/0      tcp dpt:443
-----------------------------------------------------------------------
\end{minted}

\subsection*{2.3) Third Scenario}
Last thing could be allow localhost traffic for inter-system communication and delete some dumb chains as output and forward drop\\

\begin{minted}{bash}
-----------------------------------------------------------------------
$ sudo iptables -A INPUT -i lo -j ACCEPT
$ sudo iptables -A OUTPUT -o lo -j ACCEPT
$ sudo iptables -P OUTPUT ACCEPT
$ sudo iptables -P FORWARD ACCEPT
$ sudo iptables -S
-----------------------------------------------------------------------
-P INPUT ACCEPT
-P FORWARD ACCEPT
-P OUTPUT ACCEPT
-A INPUT -p tcp -m tcp --dport 22 -j DROP
-A INPUT -p tcp -m tcp --dport 80 -j ACCEPT
-A INPUT -p tcp -m tcp --dport 443 -j ACCEPT
-A INPUT
-A INPUT -p tcp -m tcp --dport 443 -j ACCEPT
-A INPUT -i lo -j ACCEPT
-A FORWARD -p tcp -m tcp --dport 80 -j ACCEPT
-A FORWARD -p tcp -m tcp --dport 443 -j ACCEPT
-A OUTPUT -p tcp -m tcp --dport 80 -j ACCEPT
-A OUTPUT -p tcp -m tcp --dport 443 -j ACCEPT
-A OUTPUT -o lo -j ACCEPT
-----------------------------------------------------------------------
\end{minted}

\section*{3) Saving Configurations}
Since iptables configuration will be discarded after reboot it can be saved and loaded on boot via iptables-persistent.To restore a certain configuration i can save it on a .txt file and load by using the following command\\

\begin{minted}{bash}
-----------------------------------------------------------------------
$ sudo iptables-persistent save
$ sudo iptables-save > $PATH_TO_CONFIGURATION_FILE/iptables_config.txt
$ sudo iptables-restore < $PATH_TO_CONFIGURATION_FILE/iptables_config.txt
-----------------------------------------------------------------------
\end{minted}

\end{document}
