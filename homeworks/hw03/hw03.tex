\documentclass{article}
\usepackage[utf8]{inputenc}
\title{Cybersecurity - Homework \#3}
\author{Vlad Turno (1835365)}

\begin{document}

\maketitle

\section*{1) Introduction}
Age (acronym of "Actually Good Encryption") is a modern tool for file encryption that features strong security standards alongside with a simplicity of usage.
It supports both symmetric and public key encryption and the fact of being an open source software allows for transparency and community contributions.

\section*{2) Guidelines}
With AGE it is possible to perform encryption and decryption of data via both symmetric or public/private key.
The software features a key management system that allows to generate and store key pairs (both numeric or password based).
A more detailed description about functionalities and syntax can be found in the documentation pages.

\section*{3) Algorithms}

\subsection*{3.1) Symmetric Encryption}
In order to perform symmetric encryption AGE combines two different algorithms, "ChaCha20" and "Poly1305".
The first is a fast stream cipher (i won't go along describing what a stream cipher is and how it works) and the second is a message authentication code.
By combining the two it gives an authenticated encryption that guarantees the integrity and confidentiality of the data.

\subsection*{3.2) Public Key Encryption}
For public key encryption AGE uses "Curve25519" to establish a secret and secure communication over an insecure channel (to use for key exchange purpose).
To create and verify digital signatures is used a signature scheme based on elliptic curves called "Ed25519" that ensure the authenticity of messages.

\section*{4) Use Cases}
The software AGE is useful when you have to share sensitive information and you want to ensure they are secure even if accessed by other people.
Most common use case could be sharing some file via email or uploading it on a cloud storage.
It may happen with some backup files that for obvious reasons you don't want to be located ONLY on your laptop.

\end{document}
