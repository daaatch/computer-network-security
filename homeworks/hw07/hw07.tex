\documentclass{article}
\usepackage[utf8]{inputenc}
\title{Cybersecurity - Homework \#7}
\author{Vlad Turno (1835365)}
\usepackage{minted}
\usepackage{listings}
\usepackage{graphicx}

\begin{document}

\maketitle

\section*{1) Testing a connection within HTTPS}
To test a TLS connection i type into terminal the following openssl command, making sure to select the port 443 on the specified host, which is the default https one.\\

\begin{minted}{bash}
$ openssl s_client -connect google.com:443 -showcerts

CONNECTED(00000003)
depth=2 C = US, O = Google Trust Services LLC, CN = GTS Root R1
verify return:1
depth=1 C = US, O = Google Trust Services, CN = WR2
verify return:1
depth=0 CN = *.google.com
verify return:1
---
Certificate chain (...)
---
Server certificate
subject=CN = *.google.com
issuer=C = US, O = Google Trust Services, CN = WR2
---
No client certificate CA names sent
Peer signing digest: SHA256
Peer signature type: ECDSA
Server Temp Key: X25519, 253 bits
---
SSL handshake has read 6591 bytes and written 396 bytes
Verification: OK
---
New, TLSv1.3, Cipher is TLS_AES_256_GCM_SHA384
Server public key is 256 bit
Secure Renegotiation IS NOT supported
Compression: NONE
Expansion: NONE
No ALPN negotiated
Early data was not sent
Verify return code: 0 (ok)
---
\end{minted}

\section*{2) Automated testing with BASH}
By executing the following .sh file i can automate to call specified hosts from a given list, writing the outcome of the connection an the time elapsed into a text file.
\\

\lstset{
  language=BASH,                     
  numbers=left,                
  stepnumber=1,                       
  numbersep=5pt,                  
  backgroundcolor=\color{white},  
  showspaces=false,               
  showstringspaces=false,         
  showtabs=false,               
  tabsize=2,                      
  captionpos=b,                  
  breaklines=true,               
  breakatwhitespace=true,     
  title=\lstname,             
}
\lstinputlisting{tls_test.sh}

\section*{4) Session Outcome}
Now take a couple of example of TLS connections, one successful and one not.\\

\begin{minted}{bash}
$ openssl s_client -connect openssl.org:443 -showcerts

depth=2 C = US, O = Google Trust Services LLC, CN = GTS Root R1
verify return:1
depth=1 C = US, O = Google Trust Services, CN = WR3
verify return:1
depth=0 CN = openssl.org
verify return:1
CONNECTED(00000003)
---
Certificate chain
 0 s:CN = openssl.org
   i:C = US, O = Google Trust Services, CN = WR3
   a:PKEY: rsaEncryption, 2048 (bit); sigalg: RSA-SHA256
   v:NotBefore: Nov 26 17:47:59 2024 GMT; NotAfter: Feb 24 18:42:33 2025 GMT
 1 s:C = US, O = Google Trust Services, CN = WR3
   i:C = US, O = Google Trust Services LLC, CN = GTS Root R1
   a:PKEY: rsaEncryption, 2048 (bit); sigalg: RSA-SHA256
   v:NotBefore: Dec 13 09:00:00 2023 GMT; NotAfter: Feb 20 14:00:00 2029 GMT
 2 s:C = US, O = Google Trust Services LLC, CN = GTS Root R1
   i:C = BE, O = GlobalSign nv-sa, OU = Root CA, CN = GlobalSign Root CA
   a:PKEY: rsaEncryption, 4096 (bit); sigalg: RSA-SHA256
   v:NotBefore: Jun 19 00:00:42 2020 GMT; NotAfter: Jan 28 00:00:42 2028 GMT
---
Server certificate
-----BEGIN CERTIFICATE-----
MIIFMjCCBBqgAwIBAgIRAI7ivVEq4CyIEMXIcA9O6i0wDQYJKoZIhvcNAQELBQAw
OzELMAkGA1UEBhMCVVMxHjAcBgNVBAoTFUdvb2dsZSBUcnVzdCBTZXJ2aWNlczEM
(...)
/CHChV1U4aXqrGKa/NAOSJsnea3U2GmBZhNA85NS8fg3OJUO0Ke6/FryowyaEVY0
JCr2EBm6XdbM/wgIjYA6d8N5cCvCzCzzz9YiO5nqNslUvVYC0GM=
-----END CERTIFICATE-----
subject=CN = openssl.org
issuer=C = US, O = Google Trust Services, CN = WR3
---
No client certificate CA names sent
Peer signing digest: SHA256
Peer signature type: RSA-PSS
Server Temp Key: X25519, 253 bits
---
SSL handshake has read 4511 bytes and written 325 bytes
Verification: OK
---
New, TLSv1.3, Cipher is TLS_AES_256_GCM_SHA384
Server public key is 2048 bit
Secure Renegotiation IS NOT supported
Compression: NONE
Expansion: NONE
No ALPN negotiated
Early data was not sent
Verify return code: 0 (ok)
---
DONE
\end{minted}

In this first example we can read the certificate chain from server to certificate authority's root followed by verification code (verify return:1) and connection code (CONNECTED (0000003)), both successful. Then there are certificate details and server certificate (cut off because of length) and TLS handshake details such as byte read and written, session information like tls version, cipher and protocol used, server public key, protocol negotiation and so on.\\

\begin{minted}{bash}
$ openssl s_client -connect uniroma1.it:443 -showcerts

40F7559FE87F0000:error:0A000410:SSL routines:ssl3_read_bytes:sslv3 alert handshake failure:../ssl/record/rec_layer_s3.c:1605:SSL alert number 40
CONNECTED(00000003)
---
no peer certificate available
---
No client certificate CA names sent
---
SSL handshake has read 7 bytes and written 245 bytes
Verification: OK
---
New, (NONE), Cipher is (NONE)
Secure Renegotiation IS NOT supported
Compression: NONE
Expansion: NONE
No ALPN negotiated
Early data was not sent
Verify return code: 0 (ok)
---
\end{minted}

In this second example is shown how the outcome appears in case of unsuccessful connection.\\

\section*{4) Session Limits}
In examples TLS version 1.3 (2018) have been used, and there could be some compatibility issues since some clients or servers could not support it (TLS 1.0 is from 1999). Also, older versions of TLS have weak performances and this could lead to overheading while using different version. Another thing about TLS 1.3 is that, to improve security in the connections, it doesn't have renegotiation opportunities.

\section*{5) Session Timing}
So for each session here is the time that took to finalize the TLS handshake, put into visually distinct color diagram.\\

\begin{figure}[h]
  \centering
  \includegraphics[width=\textwidth]{time_elapsed.png}
  \caption{}
  \label{fig:enc}
\end{figure}

\end{document}
