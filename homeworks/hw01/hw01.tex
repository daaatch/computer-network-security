\documentclass{article}

\title{Cybersecurity - Homework \#1}
\author{Vlad Turno (1835365)}
\usepackage{tcolorbox}
\usepackage{listings}

\begin{document}

\maketitle

\section*{1) Introduction}
In cryptography, frequency analysis is a technique that studies the number of occurrences in a given ciphertext in order to get the plaintext, thanks to the fact that in written language certain characters - such as vowels - occur in certain percentages (or at least in a meaningful speech).\\This technique is efficient for decripting substitution ciphers, where the encoding of a message consists in a direct correspondence between a plaintext character and an encrypted one.\\Now consider an example of ciphertext encrypted by substitution: 

\begin{tcolorbox} [title=I.ct]
PIFFMKMQI'YRKJKPQMKDJ RKJKPXAKZQXVRKNGJMXZ'KXZKTHRKVTYRRTKAQZZ
JKPRKJKPXAKDJZKVQDRKFJMKMQIKAQTKDIFKQZKMQ'KSJORKMQIKPXAKFXVAYJ
ORK XO XZ'KMQIYKOJZKJGGKQLRYKTHRKNGJORKVXZAXZ'
\end{tcolorbox}

\section*{2) Frequency Analysis}
With a simple C program it is possible to read the file and count how many times a certain character occurs within the ciphertext and print the result (C source code following):\\

character: K occurrences: 35 frequency 20.58\%

character: R occurrences: 13 frequency 07.64\%

character: J occurrences: 13 frequency 07.64\%

character: Q occurrences: 12 frequency 07.05\%

character: X occurrences: 11 frequency 06.47\%

character: Z occurrences: 11 frequency 06.47\%

character: M occurrences: 09 frequency 05.29\%

character: A occurrences: 07 frequency 04.11\%

character: P occurrences: 06 frequency 03.52\%

character: I occurrences: 06 frequency 03.52\%

character: F occurrences: 05 frequency 02.94\%

character: ' occurrences: 05 frequency 02.94\%

character: V occurrences: 05 frequency 02.94\%

character: T occurrences: 05 frequency 02.94\%

character: O occurrences: 05 frequency 02.94\%

character: Y occurrences: 05 frequency 02.94\%

character: D occurrences: 04 frequency 02.35\%

character: G occurrences: 04 frequency 02.35\%

character:   occurrences: 03 frequency 01.74\%

character: N occurrences: 02 frequency 01.17\%

character: H occurrences: 02 frequency 01.17\%

character: S occurrences: 01 frequency 00.58\%

character: L occurrences: 01 frequency 00.58\%

\lstset{
  language=C,                     
  numbers=left,                
  stepnumber=1,                       
  numbersep=5pt,                  
  backgroundcolor=\color{white},  
  showspaces=false,               
  showstringspaces=false,         
  showtabs=false,               
  tabsize=2,                      
  captionpos=b,                  
  breaklines=true,               
  breakatwhitespace=true,     
  title=\lstname,             
}
\lstinputlisting{frequencyAnalyzer.c}

\section*{3) Decryption Attempt}
After asking the internet what is the average percentage of the A-Z and " " (space) character occurrence it is possible to operate a substitution with another simple C program:\\

character A frequency 8.17\%

character B frequency 1.49\%

character C frequency 2.78\%

character D frequency 4.25\%

character E frequency 12.70\%

character F frequency 2.23\%

character G frequency 2.02\%

character H frequency 6.09\%

character I frequency 6.97\%

character J frequency 0.15\%

character K frequency 0.77\%

character L frequency 4.03\%

character M frequency 2.41\%

character N frequency 6.75\%

character O frequency 7.51\%

character P frequency 1.93\%

character Q frequency 0.10\%

character R frequency 5.99\%

character S frequency 6.33\%

character T frequency 9.06\%

character U frequency 2.76\%

character V frequency 0.98\%

character W frequency 2.36\%

character X frequency 0.15\%

character Y frequency 1.97\%

character Z frequency 0.07\%

character " " frequency 17\%

\lstset{
  language=C,                     
  numbers=left,                
  stepnumber=1,                       
  numbersep=5pt,                  
  backgroundcolor=\color{white},  
  showspaces=false,               
  showstringspaces=false,         
  showtabs=false,               
  tabsize=2,                      
  captionpos=b,                  
  breaklines=true,               
  breakatwhitespace=true,     
  title=\lstname,             
}
\lstinputlisting{characterSubstitution.c}

\section*{4) Observations}
It's easy to note how short the text is, so the percentages are not that accurate. But there are hints very useful to adjust the aim and make a better choice about the character substitution to operate in order to get a correctly working decryption algorithm. Here is the result of first attempt decryption (sounds like a Queen song to me):\\ 

\begin{tcolorbox} [title=plaintext.txt]
BUDDY YOULRE A BOY MAKE A BNG TONCE PSAYNTL NT WHE CWREEW GOTTA BE A BNG MAT COME DAY YOU GOW MUD OT YOL VAFE YOU BNG DNCGRAFE KNFKNTL YOUR FAT ASS OIER WHE PSAFE CNTGNTLL
\end{tcolorbox}

\end{document}
